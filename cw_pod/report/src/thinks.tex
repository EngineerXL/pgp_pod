\section{Выводы}
Трассировка лучей используется в играх для обеспечения более выского уровня реализма по сравнению с традиционными способа рендеринга. Эта технология подразумевает моделирование распространения луча света, его отражение и преломление от всех поверхностей на сцене, за счёт чего и получается более реалистичесное изображение.

Полагая, что около половины лучей не будет пересекаться ни с одним полигоном и около половины лучей удвоится, сложность обратной трассировки лучей $O(k \cdot n \cdot m \cdot l)$ --- для всех $k$ шагов рекурсии вычисляется пересечение $n$ лучей с $m$ полигонами на сцене, затем вычисляются пересечения порядка $n$ теневых лучей со всеми $m$ полигонами для всех $l$ источников света. Если же на каждом шаге рекурсии количество лучей удваивается, то сложность будет $O(2^{k} \cdot n \cdot m \cdot l)$. Разбиение пространства позволяет не искать пересечение с каждым полигоном на сцене, что позволяет существенно уменьшить сложность алгоритма.

Количество лучей зависит от входных параметров $n = h \cdot w \cdot r$, где $h$ и $w$ --- высота и ширина изображения, $r$ --- количество лучей на один пиксель.

Замеры времени работы показали, что время исполнения зависит линейно от глубины рекурсии, при этом графический процессор выигрывает у центрального за счёт параллельной обработки лучей.
\pagebreak
