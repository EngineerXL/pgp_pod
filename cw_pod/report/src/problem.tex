\section{Условие}
\subsection{Цель работы}
Использование GPU для создание фотореалистической визуализации. Рендеринг полузеркальных и полупрозрачных правильных геометрических тел. Получение эффекта бесконечности. Создание анимации.

\subsection{Сцена}
Прямоугольная текстурированная поверхность (пол), над которой расположены три платоновых тела. Сверху находятся несколько источников света. На каждом ребре многогранника располагается определенное количество точечных источников света. Грани тел обладают зеркальным и прозрачным эффектом. За счет многократного переотражения лучей внутри тела, возникает эффект бесконечности.

\subsection{Общая постановка задачи}
Требуется реализовать алгоритм обратной трассировки лучей с использованием технологии CUDA. Выполнить покадровый рендеринг сцены. Для устранения эффекта «зубчатости», выполнить сглаживание (например с помощью алгоритма SSAA). Полученный набор кадров склеить в анимацию любым доступным программным обеспечением. Подобрать параметры сцены, камеры и освещения таким образом, чтобы получить наиболее красочный  результат. Провести сравнение производительности gpu и cpu (т.е. дополнительно нужно реализовать алгоритм без использования CUDA).

\subsection{Вариант 7}
На сцене должны распологаться три тела: гексаэдр, октаэдр, додекаэдр.
\pagebreak
