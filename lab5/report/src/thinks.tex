\section{Выводы}
Сортировка используется во многих алгоритмах, например она необходима для применения бинарного поиска и для построения выпуклой оболочки алгоритмом Грэхема.

Средняя сложность алгоритма $O(n)$, так как карманная сортировка основывается на идее, что все карманы будут примерно равны по размеру.

Результаты показали, что при полученной сложности центральный выигрывает у графического при больших $n$. При параллельном вычислении возрастает константа у функций поиска максимума и минимума, построения массива префиксных сумм: $O(n \cdot \log{k})$, где $k$ --- количество потоков на блок. Так же эти функции требуют дополнительную память, на выделение которой тоже тратится немало времени.
\pagebreak
