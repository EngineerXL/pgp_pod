\section{Условие}
{\bfseries Цель работы:} Ознакомление с фундаментальными алгоритмами GPU: свертка (\texttt{reduce}), сканирование (\texttt{blelloch scan}) и гистограмма (\texttt{histogram}). Реализация одной из сортировок на CUDA. Использование разделяемой и других видов памяти. Исследование производительности программы с помощью утилиты \texttt{nvprof}.

{\bfseries Вариант 7. Карманная сортировка с чет-нечет сортировкой в каждом кармане.}

Требуется реализовать карманную сортировку для чисел типа float.
Должны быть реализованы:
\begin{itemize}
    \item Алгоритм гистограммы, с использованием атомарных операций;
    \item Алгоритм свертки для любого размера, с использованием разделяемой памяти;
    \item Алгоритм сканирования для любого размера, с использованием разделяемой памяти;
    \item Алгоритм чет-нечет сортировки для карманов.
\end{itemize}
Ограничения: $n \le {10}^8$
