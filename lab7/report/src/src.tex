\section{Метод решения}
Над пространством строится регулярная сетка. С каждой ячейкой сопоставляется значение функции $u$ в точке соответствующей центру ячейки. Граничные условия реализуются через виртуальные ячейки, которые окружают рассматриваемую область. Поиск решения сводится к итерационному процессу:
$$ u^{(n+1)}_{i, j, k} = {{(u^{(n)}_{i + 1, j, k} + u^{(n)}_{i-1, j, k}) \cdot h^{-2}_x+(u^{(n)}_{i, j + 1, k} + u^{(n)}_{i, j - 1, k}) \cdot h^{-2}_y+(u^{(n)}_{i, j, k + 1} + u^{(n)}_{i, j, k - 1}) \cdot h^{-2}_z} \over {2 \cdot (h^{-2}_x + h^{-2}_y + h^{-2}_z)}}$$
Процесс останавливается, как только
$$\max_{i, j,k}|u^{(n+1)}_{i, j, k} - u^{(n)}_{i, j, k}| < \varepsilon$$

\section{Описание программы}
Межпроцессное взаимодействие осуществляется с помощью вызовов \texttt{MPI\_Sendrecv}. Рассмотрим обмен граничными значениями по координате $z$. Так как процесс сразу и отправляет, и читает, то другой процесс должен отправлять и читать в эту сторону, то есть один процесс вверх, а другой вниз, и наоборот. Реализую данный обмен в две фазы: сперва все процессы с чётным индексом по координате $z$ обращаются к нижнему процессу, а все с нечётными к верхнему. На второй фазе чётные и нечётные меняются местами.

Аналогично происходит обмен по координатам $x$ и $y$. После обмена выполняется шаг итерации на всех процессах независимо, вычисляется погрешность на шаге, происходит обмен вычисленными погрешностями через \texttt{MPI\_Allgather}, вычисление максимальной погрешности по всем процессам и переход на следующую итерацию.
\pagebreak