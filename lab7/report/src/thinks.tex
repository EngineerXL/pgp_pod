\section{Выводы}
Задача Дирихле вытекает из трёхмерного уравнения теплопроводности, которые необходимо решать при расчётах распределения температуры, например, корпуса космического корабля. Теория разностных схем широко применяется в решении дифференциальных уравнений в частных производных.

Метод Якоби позволяет найти решение задачи для разностной схемы с помощью итерационного процесса. Пусть итерационный процесс сходится за $k$ шагов, тогда сложность алгоритма $O(k \cdot {n_x} \cdot {n_y} \cdot {nz})$, где $n_x$, $n_y$, $n_z$ --- размеры сетки по соответствующим координатам. Если эти размеры равны, то сложность будет $O(k \cdot n^3)$. Метод Зейделя сходится гораздо быстрее метода Якоби, однако его труднее распараллелить.

Результаты показали, что при маленькой сетке большое количество процессов только замедляем программу из-за затрат на синхронизацию. На большой сетке видно уменьшение времени исполнения вплоть до 6 процессов, 6-12 процессов примерно одинаково. Так как мой процессор имеет 6 физических ядер и 12 поток, то при запуске на 16 процессах программа сильно замедляется, из-за того, что часть процессов находится в блокировке.
\pagebreak
